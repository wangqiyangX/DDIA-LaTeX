\documentclass{book}
\usepackage[
  colorlinks=true,       % 启用颜色链接
  linkcolor=black,       % 内部链接(如目录)的颜色
  citecolor=red,         % 引用链接的颜色
  filecolor=magenta,     % 本地文件链接的颜色
  urlcolor=cyan          % 外部 URL 链接的颜色
]{hyperref}
\usepackage{graphicx}
\usepackage[utf8]{inputenc}
\usepackage{fontspec}  % 使用系统字体
\usepackage{xeCJK}     % 支持中文
\usepackage{listings}
\usepackage{mdframed}
\usepackage{xcolor}

\linespread{1.2}  % 设置行距为常规印刷
\lstset{
  backgroundcolor=\color{lightgray}, % 设置背景颜色
  basicstyle=\ttfamily\small,         % 设置代码字体样式
  keywordstyle=\color{blue},          % 设置关键词样式
  commentstyle=\color{green},         % 设置注释样式
  stringstyle=\color{red},            % 设置字符串样式
  frame=none,                         % 设置边框
  breaklines=true,                     % 自动换行
  xleftmargin=1em, xrightmargin=1em,   % 左右缩进留白(更美观)
}

\begin{document}
\tableofcontents

\part{数据系统基础}
\label{part:part-1}
本书前四章介绍了数据系统底层的基础概念,无论是在单台机器上运行的单点数据系统,还是分布在多台机器上的分布式数据系统都适用。

\begin{enumerate}
  \item \textbf{\autoref{ch:ch1}} 将介绍本书使用的术语和方法。\textbf{可靠性,可伸缩性和可维护性},这些词汇到底意味着什么?如何实现这些目标?
  \item \textbf{\autoref{ch:ch2}} 将对几种不同的\textbf{数据模型和查询语言}进行比较。从程序员的角度看,这是数据库之间最明显的区别。不同的数据模型适用于不同的应用场景。
  \item \textbf{\autoref{ch:ch3}} 将深入\textbf{存储引擎}内部,研究数据库如何在磁盘上摆放数据。不同的存储引擎针对不同的负载进行优化,选择合适的存储引擎对系统性能有巨大影响。
  \item \textbf{\autoref{ch:ch4}} 将对几种不同的\textbf{数据编码}进行比较。特别研究了这些格式在应用需求经常变化、模式需要随时间演变的环境中表现如何。
\end{enumerate}

第二部分将专门讨论在\textbf{分布式数据系统}中特有的问题。

\newpage
\begin{figure}
  \centering
  \includegraphics[width=\textwidth]{img/ch1.jpg}
  \label{fig:ch1}
\end{figure}

\chapter{可靠性、可伸缩性和可维护性}
\label{ch:ch1}
\begin{quote}
  互联网做得太棒了,以至于大多数人将它看作像太平洋这样的自然资源,而不是什么人工产物。上一次出现这种大规模且无差错的技术,你还记得是什么时候吗?\\
  \textit{—— \href{http://www.drdobbs.com/architecture-and-design/interview-with-alan-kay/240003442}{艾伦・凯} 在接受 Dobb 博士杂志采访时说(2012 年)}
\end{quote}

现今很多应用程序都是 \textbf{数据密集型(data-intensive)} 的,而非 \textbf{计算密集型(compute-intensive)} 的。因此 CPU 很少成为这类应用的瓶颈,更大的问题通常来自数据量、数据复杂性、以及数据的变更速度。

数据密集型应用通常由标准组件构建而成,标准组件提供了很多通用的功能;例如,许多应用程序都需要:

\begin{itemize}
  \item 存储数据,以便自己或其他应用程序之后能再次找到(\textit{数据库,即 databases})
  \item 记住开销昂贵操作的结果,加快读取速度(\textit{缓存,即 caches})
  \item 允许用户按关键字搜索数据,或以各种方式对数据进行过滤(\textit{搜索索引,即 search indexes})
  \item 向其他进程发送消息,进行异步处理(\textit{流处理,即 stream processing})
  \item 定期处理累积的大批量数据(\textit{批处理,即 batch processing})
\end{itemize}

如果这些功能听上去平淡无奇,那是因为这些 \textbf{数据系统(data system)} 是非常成功的抽象:我们一直不假思索地使用它们并习以为常。绝大多数工程师不会幻想从零开始编写存储引擎,因为在开发应用时,数据库已经是足够完美的工具了。

但现实没有这么简单。不同的应用有着不同的需求,因而数据库系统也是百花齐放,有着各式各样的特性。实现缓存有很多种手段,创建搜索索引也有好几种方法,诸如此类。因此在开发应用前,我们依然有必要先弄清楚最适合手头工作的工具和方法。而且当单个工具解决不了你的问题时,组合使用这些工具可能还是有些难度的。

本书将是一趟关于数据系统原理、实践与应用的旅程,并讲述了设计数据密集型应用的方法。我们将探索不同工具之间的共性与特性,以及各自的实现原理。

本章将从我们所要实现的基础目标开始:可靠、可伸缩、可维护的数据系统。我们将澄清这些词语的含义,概述考量这些目标的方法。并回顾一些后续章节所需的基础知识。在接下来的章节中我们将抽丝剥茧,研究设计数据密集型应用时可能遇到的设计决策。

\section{关于数据系统的思考}
我们通常认为,数据库、消息队列、缓存等工具分属于几个差异显著的类别。虽然数据库和消息队列表面上有一些相似性 —— 它们都会存储一段时间的数据 —— 但它们有迥然不同的访问模式,这意味着迥异的性能特征和实现手段。

那我们为什么要把这些东西放在 \textbf{数据系统(data system)} 的总称之下混为一谈呢?

近些年来,出现了许多新的数据存储工具与数据处理工具。它们针对不同应用场景进行优化,因此不再适合生硬地归入传统类别\cite{stonebraker2005}。类别之间的界限变得越来越模糊,例如:数据存储可以被当成消息队列用(Redis),消息队列则带有类似数据库的持久保证(Apache Kafka)。

其次,越来越多的应用程序有着各种严格而广泛的要求,单个工具不足以满足所有的数据处理和存储需求。取而代之的是,总体工作被拆分成一系列能被单个工具高效完成的任务,并通过应用代码将它们缝合起来。

例如,如果将缓存(应用管理的缓存层,Memcached 或同类产品)和全文搜索(全文搜索服务器,例如 Elasticsearch 或 Solr)功能从主数据库剥离出来,那么使缓存 / 索引与主数据库保持同步通常是应用代码的责任。\href{img/fig1-1.png}{图 1-1} 给出了这种架构可能的样子(细节将在后面的章节中详细介绍)。

\begin{figure}[h]
  \centering
  \includegraphics[width=0.8\textwidth]{img/fig1-1.png}
  \caption{一个可能的组合使用多个组件的数据系统架构}
\end{figure}

当你将多个工具组合在一起提供服务时,服务的接口或 \textbf{应用程序编程接口(API, Application Programming Interface)} 通常向客户端隐藏这些实现细节。现在,你基本上已经使用较小的通用组件创建了一个全新的、专用的数据系统。这个新的复合数据系统可能会提供特定的保证,例如:缓存在写入时会作废或更新,以便外部客户端获取一致的结果。现在你不仅是应用程序开发人员,还是数据系统设计人员了。

设计数据系统或服务时可能会遇到很多棘手的问题,例如:当系统出问题时,如何确保数据的正确性和完整性?当部分系统退化降级时,如何为客户提供始终如一的良好性能?当负载增加时,如何扩容应对?什么样的 API 才是好的 API?

影响数据系统设计的因素很多,包括参与人员的技能和经验、历史遗留问题、系统路径依赖、交付时限、公司的风险容忍度、监管约束等,这些因素都需要具体问题具体分析。

本书着重讨论三个在大多数软件系统中都很重要的问题:

\begin{itemize}
  \item \textbf{可靠性(Reliability)}

        系统在 \textbf{困境}(adversity,比如硬件故障、软件故障、人为错误)中仍可正常工作(正确完成功能,并能达到期望的性能水准)。请参阅 “\hyperref[sec:reliability]{可靠性}”。

  \item \textbf{可伸缩性(Scalability)}

        有合理的办法应对系统的增长(数据量、流量、复杂性)。请参阅 “\hyperref[sec:scalability]{可伸缩性}”。

  \item \textbf{可维护性(Maintainability)}

        许多不同的人(工程师、运维)在不同的生命周期,都能高效地在系统上工作(使系统保持现有行为,并适应新的应用场景)。请参阅 “\hyperref[sec:maintainability]{可维护性}”。
\end{itemize}

人们经常追求这些词汇,却没有清楚理解它们到底意味着什么。为了工程的严谨性,本章的剩余部分将探讨可靠性、可伸缩性和可维护性的含义。为实现这些目标而使用的各种技术,架构和算法将在后续的章节中研究。

\section{可靠性}
\label{sec:reliability}

人们对于一个东西是否可靠,都有一个直观的想法。人们对可靠软件的典型期望包括:

\begin{itemize}

  \item 应用程序表现出用户所期望的功能。

  \item 允许用户犯错,允许用户以出乎意料的方式使用软件。

  \item 在预期的负载和数据量下,性能满足要求。

  \item 系统能防止未经授权的访问和滥用。

\end{itemize}

如果所有这些在一起意味着 “正确工作”,那么可以把可靠性粗略理解为 “即使出现问题,也能继续正确工作”。

造成错误的原因叫做 \textbf{故障(fault)},能预料并应对故障的系统特性可称为 \textbf{容错(fault-tolerant)} 或 \textbf{回弹性(resilient)}。“\textbf{容错}” 一词可能会产生误导,因为它暗示着系统可以容忍所有可能的错误,但在实际中这是不可能的。比方说,如果整个地球(及其上的所有服务器)都被黑洞吞噬了,想要容忍这种错误,需要把网络托管到太空中 —— 这种预算能不能批准就祝你好运了。所以在讨论容错时,只有谈论特定类型的错误才有意义。

注意 \textbf{故障(fault)} 不同于 \textbf{失效(failure)}【2】。\textbf{故障} 通常定义为系统的一部分状态偏离其标准,而 \textbf{失效} 则是系统作为一个整体停止向用户提供服务。故障的概率不可能降到零,因此最好设计容错机制以防因 \textbf{故障} 而导致 \textbf{失效}。本书中我们将介绍几种用不可靠的部件构建可靠系统的技术。

反直觉的是,在这类容错系统中,通过故意触发来 \textbf{提高} 故障率是有意义的,例如:在没有警告的情况下随机地杀死单个进程。许多高危漏洞实际上是由糟糕的错误处理导致的【3】,因此我们可以通过故意引发故障来确保容错机制不断运行并接受考验,从而提高故障自然发生时系统能正确处理的信心。Netflix 公司的 \textit{Chaos Monkey}【4】就是这种方法的一个例子。

尽管比起 \textbf{阻止错误(prevent error)},我们通常更倾向于 \textbf{容忍错误}。但也有 \textbf{预防胜于治疗} 的情况(比如不存在治疗方法时)。安全问题就属于这种情况。例如,如果攻击者破坏了系统,并获取了敏感数据,这种事是撤销不了的。但本书主要讨论的是可以恢复的故障种类,正如下面几节所述。

\subsection{硬件故障}

当想到系统失效的原因时,\textbf{硬件故障(hardware faults)} 总会第一个进入脑海。硬盘崩溃、内存出错、机房断电、有人拔错网线…… 任何与大型数据中心打过交道的人都会告诉你:一旦你拥有很多机器,这些事情 \textbf{总} 会发生!

据报道称,硬盘的 \textbf{平均无故障时间(MTTF, mean time to failure)} 约为 10 到 50 年【5】【6】。因此从数学期望上讲,在拥有 10000 个磁盘的存储集群上,平均每天会有 1 个磁盘出故障。

为了减少系统的故障率,第一反应通常都是增加单个硬件的冗余度,例如:磁盘可以组建 RAID,服务器可能有双路电源和热插拔 CPU,数据中心可能有电池和柴油发电机作为后备电源,某个组件挂掉时冗余组件可以立刻接管。这种方法虽然不能完全防止由硬件问题导致的系统失效,但它简单易懂,通常也足以让机器不间断运行很多年。

直到最近,硬件冗余对于大多数应用来说已经足够了,它使单台机器完全失效变得相当罕见。只要你能快速地把备份恢复到新机器上,故障停机时间对大多数应用而言都算不上灾难性的。只有少量高可用性至关重要的应用才会要求有多套硬件冗余。

但是随着数据量和应用计算需求的增加,越来越多的应用开始大量使用机器,这会相应地增加硬件故障率。此外,在类似亚马逊 AWS(Amazon Web Services)的一些云服务平台上,虚拟机实例不可用却没有任何警告也是很常见的【7】,因为云平台的设计就是优先考虑 \textbf{灵活性(flexibility)} 和 \textbf{弹性(elasticity)}\footnote{在\nameref{sec:应对负载的方法}一节定义},而不是单机可靠性。

如果在硬件冗余的基础上进一步引入软件容错机制,那么系统在容忍整个(单台)机器故障的道路上就更进一步了。这样的系统也有运维上的便利,例如:如果需要重启机器(例如应用操作系统安全补丁),单服务器系统就需要计划停机。而允许机器失效的系统则可以一次修复一个节点,无需整个系统停机。

\subsection{软件错误}

我们通常认为硬件故障是随机的、相互独立的:一台机器的磁盘失效并不意味着另一台机器的磁盘也会失效。虽然大量硬件组件之间可能存在微弱的相关性(例如服务器机架的温度等共同的原因),但同时发生故障也是极为罕见的。

另一类错误是内部的 \textbf{系统性错误(systematic error)}【8】。这类错误难以预料,而且因为是跨节点相关的,所以比起不相关的硬件故障往往可能造成更多的 \textbf{系统失效}【5】。例子包括:

\begin{itemize}

  \item 接受特定的错误输入,便导致所有应用服务器实例崩溃的 BUG。例如 2012 年 6 月 30 日的闰秒,由于 Linux 内核中的一个错误【9】,许多应用同时挂掉了。

  \item 失控进程会用尽一些共享资源,包括 CPU 时间、内存、磁盘空间或网络带宽。

  \item 系统依赖的服务变慢,没有响应,或者开始返回错误的响应。

  \item 级联故障,一个组件中的小故障触发另一个组件中的故障,进而触发更多的故障【10】。

\end{itemize}

导致这类软件故障的 BUG 通常会潜伏很长时间,直到被异常情况触发为止。这种情况意味着软件对其环境做出了某种假设 —— 虽然这种假设通常来说是正确的,但由于某种原因最后不再成立了【11】。

虽然软件中的系统性故障没有速效药,但我们还是有很多小办法,例如:仔细考虑系统中的假设和交互;彻底的测试;进程隔离;允许进程崩溃并重启;测量、监控并分析生产环境中的系统行为。如果系统能够提供一些保证(例如在一个消息队列中,进入与发出的消息数量相等),那么系统就可以在运行时不断自检,并在出现 \textbf{差异(discrepancy)} 时报警【12】。

\subsection{人为错误}

设计并构建了软件系统的工程师是人类,维持系统运行的运维也是人类。即使他们怀有最大的善意,人类也是不可靠的。举个例子,一项关于大型互联网服务的研究发现,运维配置错误是导致服务中断的首要原因,而硬件故障(服务器或网络)仅导致了 10-25\% 的服务中断【13】。

尽管人类不可靠,但怎么做才能让系统变得可靠?最好的系统会组合使用以下几种办法:

\begin{itemize}

  \item 以最小化犯错机会的方式设计系统。例如,精心设计的抽象、API 和管理后台使做对事情更容易,搞砸事情更困难。但如果接口限制太多,人们就会忽略它们的好处而想办法绕开。很难正确把握这种微妙的平衡。

  \item 将人们最容易犯错的地方与可能导致失效的地方 \textbf{解耦(decouple)}。特别是提供一个功能齐全的非生产环境 \textbf{沙箱(sandbox)},使人们可以在不影响真实用户的情况下,使用真实数据安全地探索和实验。

  \item 在各个层次进行彻底的测试【3】,从单元测试、全系统集成测试到手动测试。自动化测试易于理解,已经被广泛使用,特别适合用来覆盖正常情况中少见的 \textbf{边缘场景(corner case)}。

  \item 允许从人为错误中简单快速地恢复,以最大限度地减少失效情况带来的影响。例如,快速回滚配置变更,分批发布新代码(以便任何意外错误只影响一小部分用户),并提供数据重算工具(以备旧的计算出错)。

  \item 配置详细和明确的监控,比如性能指标和错误率。在其他工程学科中这指的是 \textbf{遥测(telemetry)}(一旦火箭离开了地面,遥测技术对于跟踪发生的事情和理解失败是至关重要的)。监控可以向我们发出预警信号,并允许我们检查是否有任何地方违反了假设和约束。当出现问题时,指标数据对于问题诊断是非常宝贵的。

  \item 良好的管理实践与充分的培训 —— 一个复杂而重要的方面,但超出了本书的范围。

\end{itemize}

\subsection{可靠性有多重要?}

可靠性不仅仅是针对核电站和空中交通管制软件而言,我们也期望更多平凡的应用能可靠地运行。商务应用中的错误会导致生产力损失(也许数据报告不完整还会有法律风险),而电商网站的中断则可能会导致收入和声誉的巨大损失。

即使在 “非关键” 应用中,我们也对用户负有责任。试想一位家长把所有的照片和孩子的视频储存在你的照片应用里【15】。如果数据库突然损坏,他们会感觉如何?他们可能会知道如何从备份恢复吗?

在某些情况下,我们可能会选择牺牲可靠性来降低开发成本(例如为未经证实的市场开发产品原型)或运营成本(例如利润率极低的服务),但我们偷工减料时,应该清楚意识到自己在做什么。

\section{可伸缩性}
\label{sec:scalability}

系统今天能可靠运行,并不意味未来也能可靠运行。服务 \textbf{降级(degradation)} 的一个常见原因是负载增加,例如:系统负载已经从一万个并发用户增长到十万个并发用户,或者从一百万增长到一千万。也许现在处理的数据量级要比过去大得多。

\textbf{可伸缩性(Scalability)} 是用来描述系统应对负载增长能力的术语。但是请注意,这不是贴在系统上的一维标签:说 “X 可伸缩” 或 “Y 不可伸缩” 是没有任何意义的。相反,讨论可伸缩性意味着考虑诸如 “如果系统以特定方式增长,有什么选项可以应对增长?” 和 “如何增加计算资源来处理额外的负载?” 等问题。

\subsection{描述负载}

在讨论增长问题(如果负载加倍会发生什么?)前,首先要能简要描述系统的当前负载。负载可以用一些称为 \textbf{负载参数(load parameters)} 的数字来描述。参数的最佳选择取决于系统架构,它可能是每秒向 Web 服务器发出的请求、数据库中的读写比率、聊天室中同时活跃的用户数量、缓存命中率或其他东西。除此之外,也许平均情况对你很重要,也许你的瓶颈是少数极端场景。

为了使这个概念更加具体,我们以推特在 2012 年 11 月发布的数据【16】为例。推特的两个主要业务是:

\begin{itemize}
  \item
        发布推文

        用户可以向其粉丝发布新消息(平均 4.6k 请求 / 秒,峰值超过 12k 请求 /秒)。
  \item
        主页时间线

        用户可以查阅他们关注的人发布的推文(300k 请求 / 秒)。
\end{itemize}

处理每秒 12,000 次写入(发推文的速率峰值)还是很简单的。然而推特的伸缩性挑战并不是主要来自推特量,而是来自 \textbf{扇出(fan-out)}\footnote{扇出:从电子工程学中借用的术语,它描述了输入连接到另一个门输出的逻辑门数量。输出需要提供足够的电流来驱动所有连接的输入。在事务处理系统中,我们使用它来描述为了服务一个传入请求而需要执行其他服务的请求数量。}—— 每个用户关注了很多人,也被很多人关注。

大体上讲,这一对操作有两种实现方式。

\begin{enumerate}
  \item 发布推文时,只需将新推文插入全局推文集合即可。当一个用户请求自己的主页时间线时,首先查找他关注的所有人,查询这些被关注用户发布的推文并按时间顺序合并。在如 \autoref{fig:fig1-2} 所示的关系型数据库中,可以编写这样的查询:
        \begin{lstlisting}[language=sql]
          SELECT tweets., users.
          FROM tweets
          JOIN users ON tweets.sender_id = users.id
          JOIN follows ON follows.followee_id = users.id
          WHERE follows.follower_id = current_user
        \end{lstlisting}
        \begin{figure}
          \centering
          \includegraphics[width=0.5\textwidth]{img/fig1-2.png}
          \caption{图 1-2 推特主页时间线的关系型模式简单实现}
          \label{fig:fig1-2}
        \end{figure}
  \item 为每个用户的主页时间线维护一个缓存,就像每个用户的推文收件箱(\autoref{fig:fig1-3})。当一个用户发布推文时,查找所有关注该用户的人,并将新的推文插入到每个主页时间线缓存中。因此读取主页时间线的请求开销很小,因为结果已经提前计算好了。
\end{enumerate}

2. 为每个用户的主页时间线维护一个缓存,就像每个用户的推文收件箱(\autoref{fig:fig1-3})。当一个用户发布推文时,查找所有关注该用户的人,并将新的推文插入到每个主页时间线缓存中。因此读取主页时间线的请求开销很小,因为结果已经提前计算好了。

\begin{figure}[htbp]

  \centering

  \includegraphics[width=0.8\textwidth]{img/fig1-3.png}

  \caption{用于分发推特至关注者的数据流水线,2012 年 11 月的负载参数【16】}

  \label{fig:fig1-3}

\end{figure}

推特的第一个版本使用了方法 1,但系统很难跟上主页时间线查询的负载。所以公司转向了方法 2,方法 2 的效果更好,因为发推频率比查询主页时间线的频率几乎低了两个数量级,所以在这种情况下,最好在写入时做更多的工作,而在读取时做更少的工作。

然而方法 2 的缺点是,发推现在需要大量的额外工作。平均来说,一条推文会发往约 75 个关注者,所以每秒 4.6k 的发推写入,变成了对主页时间线缓存每秒 345k 的写入。但这个平均值隐藏了用户粉丝数差异巨大这一现实,一些用户有超过 3000 万的粉丝,这意味着一条推文就可能会导致主页时间线缓存的 3000 万次写入!及时完成这种操作是一个巨大的挑战 —— 推特尝试在 5 秒内向粉丝发送推文。

在推特的例子中,每个用户粉丝数的分布(可能按这些用户的发推频率来加权)是探讨可伸缩性的一个关键负载参数,因为它决定了扇出负载。你的应用程序可能具有非常不同的特征,但可以采用相似的原则来考虑它的负载。

推特轶事的最终转折:现在已经稳健地实现了方法 2,推特逐步转向了两种方法的混合。大多数用户发的推文会被扇出写入其粉丝主页时间线缓存中。但是少数拥有海量粉丝的用户(即名流)会被排除在外。当用户读取主页时间线时,分别地获取出该用户所关注的每位名流的推文,再与用户的主页时间线缓存合并,如方法 1 所示。这种混合方法能始终如一地提供良好性能。在 \autoref{ch12} 中我们将重新讨论这个例子,这在覆盖更多技术层面之后。

\subsection{描述性能}

一旦系统的负载被描述好,就可以研究当负载增加会发生什么。我们可以从两种角度来看:

\begin{itemize}

  \item 增加负载参数并保持系统资源(CPU、内存、网络带宽等)不变时,系统性能将受到什么影响?

  \item 增加负载参数并希望保持性能不变时,需要增加多少系统资源?

\end{itemize}

这两个问题都需要性能数据,所以让我们简单地看一下如何描述系统性能。

对于 Hadoop 这样的批处理系统,通常关心的是 \textbf{吞吐量(throughput)},即每秒可以处理的记录数量,或者在特定规模数据集上运行作业的总时间。对于在线系统,通常更重要的是服务的 \textbf{响应时间(response time)},即客户端发送请求到接收响应之间的时间。

\begin{mdframed}
  \textbf{延迟和响应时间}

  \textbf{延迟(latency)} 和 \textbf{响应时间(response time)} 经常用作同义词,但实际上它们并不一样。响应时间是客户所看到的,除了实际处理请求的时间(\textbf{服务时间(service time)})之外,还包括网络延迟和排队延迟。延迟是某个请求等待处理的 \textbf{持续时长},在此期间它处于 \textbf{休眠(latent)} 状态,并等待服务【17】。

\end{mdframed}

即使不断重复发送同样的请求,每次得到的响应时间也都会略有不同。现实世界的系统会处理各式各样的请求,响应时间可能会有很大差异。因此我们需要将响应时间视为一个可以测量的数值 \textbf{分布(distribution)},而不是单个数值。

\subsection{延迟和响应时间}

\subsection{实践中的百分位点}

\subsection{应对负载的方法}
\label{sec:应对负载的方法}

\section{可维护性}

\begin{thebibliography}{99}

  \bibitem{stonebraker2005} Michael Stonebraker and Uğur Çetintemel: ``\textit{`One Size Fits All': An Idea Whose Time Has Come and Gone},'' at \textit{21st International Conference on Data Engineering} (ICDE), April 2005. \url{https://cs.brown.edu/~ugur/fits_all.pdf}

  \bibitem{heimerdinger1992} Walter L. Heimerdinger and Charles B. Weinstock: ``\textit{A Conceptual Framework for System Fault Tolerance},'' Technical Report CMU/SEI-92-TR-033, Software Engineering Institute, Carnegie Mellon University, October 1992. \url{https://resources.sei.cmu.edu/asset_files/TechnicalReport/1992_005_001_16112.pdf}

  \bibitem{yuan2014} Ding Yuan, Yu Luo, Xin Zhuang, et al.: ``\textit{Simple Testing Can Prevent Most Critical Failures: An Analysis of Production Failures in Distributed Data-Intensive Systems},'' at \textit{11th USENIX Symposium on Operating Systems Design and Implementation} (OSDI), October 2014. \url{https://www.usenix.org/system/files/conference/osdi14/osdi14-paper-yuan.pdf}

  \bibitem{izrailevsky2011} Yury Izrailevsky and Ariel Tseitlin: ``\textit{The Netflix Simian Army},'' \textit{netflixtechblog.com}, July 19, 2011. \url{https://netflixtechblog.com/the-netflix-simian-army-16e57fbab116}

  \bibitem{ford2010} Daniel Ford, François Labelle, Florentina I. Popovici, et al.: ``\textit{Availability in Globally Distributed Storage Systems},'' at \textit{9th USENIX Symposium on Operating Systems Design and Implementation} (OSDI), October 2010. \url{http://research.google.com/pubs/archive/36737.pdf}

  \bibitem{beach2014} Brian Beach: ``\textit{Hard Drive Reliability Update – Sep 2014},'' \textit{backblaze.com}, September 23, 2014. \url{https://www.backblaze.com/blog/hard-drive-reliability-update-september-2014/}

  \bibitem{voss2012} Laurie Voss: ``\textit{AWS: The Good, the Bad and the Ugly},'' \textit{blog.awe.sm}, December 18, 2012. \url{https://web.archive.org/web/20160429075023/http://blog.awe.sm/2012/12/18/aws-the-good-the-bad-and-the-ugly/}

  \bibitem{gunawi2014} Haryadi S. Gunawi, Mingzhe Hao, Tanakorn Leesatapornwongsa, et al.: ``\textit{What Bugs Live in the Cloud?},'' at \textit{5th ACM Symposium on Cloud Computing} (SoCC), November 2014. \url{http://ucare.cs.uchicago.edu/pdf/socc14-cbs.pdf} [doi:10.1145/2670979.2670986]

  \bibitem{minar2012} Nelson Minar: ``\textit{Leap Second Crashes Half the Internet},'' \textit{somebits.com}, July 3, 2012. \url{http://www.somebits.com/weblog/tech/bad/leap-second-2012.html}

  \bibitem{aws2011} Amazon Web Services: ``\textit{Summary of the Amazon EC2 and Amazon RDS Service Disruption in the US East Region},'' \textit{aws.amazon.com}, April 29, 2011. \url{http://aws.amazon.com/message/65648/}

  \bibitem{cook2000} Richard I. Cook: ``\textit{How Complex Systems Fail},'' Cognitive Technologies Laboratory, April 2000. \url{https://www.adaptivecapacitylabs.com/HowComplexSystemsFail.pdf}

  \bibitem{kreps2012} Jay Kreps: ``\textit{Getting Real About Distributed System Reliability},'' \textit{blog.empathybox.com}, March 19, 2012. \url{http://blog.empathybox.com/post/19574936361/getting-real-about-distributed-system-reliability}

  \bibitem{oppenheimer2003} David Oppenheimer, Archana Ganapathi, and David A. Patterson: ``\textit{Why Do Internet Services Fail, and What Can Be Done About It?},'' at \textit{4th USENIX Symposium on Internet Technologies and Systems} (USITS), March 2003. \url{http://static.usenix.org/legacy/events/usits03/tech/full_papers/oppenheimer/oppenheimer.pdf}

  \bibitem{marz2013} Nathan Marz: ``\textit{Principles of Software Engineering, Part 1},'' \textit{nathanmarz.com}, April 2, 2013. \url{http://nathanmarz.com/blog/principles-of-software-engineering-part-1.html}

  \bibitem{jurewitz2013} Michael Jurewitz: ``\textit{The Human Impact of Bugs},'' \textit{jury.me}, March 15, 2013. \url{http://jury.me/blog/2013/3/14/the-human-impact-of-bugs}

  \bibitem{krikorian2012} Raffi Krikorian: ``\textit{Timelines at Scale},'' at \textit{QCon San Francisco}, November 2012. \url{http://www.infoq.com/presentations/Twitter-Timeline-Scalability}

  \bibitem{fowler2002} Martin Fowler: \textit{Patterns of Enterprise Application Architecture}. Addison Wesley, 2002. ISBN: 978-0-321-12742-6

  \bibitem{sommers2014} Kelly Sommers: ``\textit{After all that run around, what caused 500ms disk latency even when we replaced physical server?},'' \textit{twitter.com}, November 13, 2014. \url{https://twitter.com/kellabyte/status/532930540777635840}

  \bibitem{decandia2007} Giuseppe DeCandia, Deniz Hastorun, Madan Jampani, et al.: ``\textit{Dynamo: Amazon's Highly Available Key-Value Store},'' at \textit{21st ACM Symposium on Operating Systems Principles} (SOSP), October 2007. \url{http://www.allthingsdistributed.com/files/amazon-dynamo-sosp2007.pdf}

  \bibitem{linden2006} Greg Linden: ``\textit{Make Data Useful},'' slides from presentation at Stanford University Data Mining class (CS345), December 2006. \url{http://glinden.blogspot.co.uk/2006/12/slides-from-my-talk-at-stanford.html}

  \bibitem{everts2014} Tammy Everts: ``\textit{The Real Cost of Slow Time vs Downtime},'' \textit{slideshare.net}, November 5, 2014. \url{https://www.slideshare.net/Radware/radware-cmg2014-tammyevertsslowtimevsdowntime}

  \bibitem{brutlag2009} Jake Brutlag: ``\textit{Speed Matters},'' \textit{ai.googleblog.com}, June 23, 2009. \url{https://ai.googleblog.com/2009/06/speed-matters.html}

  \bibitem{treat2015} Tyler Treat: ``\textit{Everything You Know About Latency Is Wrong},'' \textit{bravenewgeek.com}, December 12, 2015. \url{http://bravenewgeek.com/everything-you-know-about-latency-is-wrong/}

  \bibitem{dean2013} Jeffrey Dean and Luiz André Barroso: ``\textit{The Tail at Scale},'' \textit{Communications of the ACM}, volume 56, number 2, pages 74–80, February 2013. [doi:10.1145/2408776.2408794] \url{http://cacm.acm.org/magazines/2013/2/160173-the-tail-at-scale/fulltext}

  \bibitem{cormode2009} Graham Cormode, Vladislav Shkapenyuk, Divesh Srivastava, and Bojian Xu: ``\textit{Forward Decay: A Practical Time Decay Model for Streaming Systems},'' at \textit{25th IEEE International Conference on Data Engineering} (ICDE), March 2009. \url{http://dimacs.rutgers.edu/~graham/pubs/papers/fwddecay.pdf}

  \bibitem{dunning2014} Ted Dunning and Otmar Ertl: ``\textit{Computing Extremely Accurate Quantiles Using t-Digests},'' \textit{github.com}, March 2014. \url{https://github.com/tdunning/t-digest}

  \bibitem{tene2014} Gil Tene: ``\textit{HdrHistogram},'' \textit{hdrhistogram.org}. \url{http://www.hdrhistogram.org/}

  \bibitem{schwartz2016} Baron Schwartz: ``\textit{Why Percentiles Don’t Work the Way You Think},'' \textit{solarwinds.com}, November 18, 2016. \url{https://orangematter.solarwinds.com/2016/11/18/why-percentiles-dont-work-the-way-you-think/}

  \bibitem{hamilton2007} James Hamilton: ``\textit{On Designing and Deploying Internet-Scale Services},'' at \textit{21st Large Installation System Administration Conference} (LISA), November 2007. \url{https://www.usenix.org/legacy/events/lisa07/tech/full_papers/hamilton/hamilton.pdf}

  \bibitem{foote1997} Brian Foote and Joseph Yoder: ``\textit{Big Ball of Mud},'' at \textit{4th Conference on Pattern Languages of Programs} (PLoP), September 1997. \url{http://www.laputan.org/pub/foote/mud.pdf}

  \bibitem{brooks1995} Frederick P. Brooks: ``No Silver Bullet – Essence and Accident in Software Engineering,'' in \textit{The Mythical Man-Month}, Anniversary edition, Addison-Wesley, 1995. ISBN: 978-0-201-83595-3

  \bibitem{moseley2006} Ben Moseley and Peter Marks: ``\textit{Out of the Tar Pit},'' at \textit{BCS Software Practice Advancement} (SPA), 2006. \url{https://curtclifton.net/papers/MoseleyMarks06a.pdf}

  \bibitem{hickey2011} Rich Hickey: ``\textit{Simple Made Easy},'' at \textit{Strange Loop}, September 2011. \url{http://www.infoq.com/presentations/Simple-Made-Easy}

  \bibitem{pei2008} Hongyu Pei Breivold, Ivica Crnkovic, and Peter J. Eriksson: ``\textit{Analyzing Software Evolvability},'' at \textit{32nd Annual IEEE International Computer Software and Applications Conference} (COMPSAC), July 2008. [doi:10.1109/COMPSAC.2008.50] \url{http://www.es.mdh.se/pdf_publications/1251.pdf}

\end{thebibliography}

\newpage
\begin{figure}
  \centering
  \includegraphics[width=\textwidth]{img/ch2.png}
  \label{fig:ch2}
\end{figure}

\chapter{数据模型与查询语言}
\label{ch:ch2}
\begin{quote}
    语言的边界就是思想的边界。
    \textit{—— 路德维奇・维特根斯坦,《逻辑哲学》(1922)}
\end{quote}

数据模型可能是软件开发中最重要的部分了,因为它们的影响如此深远:不仅仅影响着软件的编写方式,而且影响着我们的 \textbf{解题思路}。

多数应用使用层层叠加的数据模型构建。对于每层数据模型的关键问题是:它是如何用低一层数据模型来 \textbf{表示} 的?例如:

\begin{enumerate}
    \item 作为一名应用开发人员,你观察现实世界(里面有人员、组织、货物、行为、资金流向、传感器等),并采用对象或数据结构,以及操控那些数据结构的 API 来进行建模。那些结构通常是特定于应用程序的。
    \item 当要存储那些数据结构时,你可以利用通用数据模型来表示它们,如 JSON 或 XML 文档、关系数据库中的表或图模型。
    \item 数据库软件的工程师选定如何以内存、磁盘或网络上的字节来表示 JSON / XML/ 关系 / 图数据。这类表示形式使数据有可能以各种方式来查询,搜索,操纵和处理。
    \item 在更低的层次上,硬件工程师已经想出了使用电流、光脉冲、磁场或者其他东西来表示字节的方法。
\end{enumerate}

一个复杂的应用程序可能会有更多的中间层次,比如基于 API 的 API,不过基本思想仍然是一样的:每个层都通过提供一个明确的数据模型来隐藏更低层次中的复杂性。这些抽象允许不同的人群有效地协作(例如数据库厂商的工程师和使用数据库的应用程序开发人员)。

数据模型种类繁多,每个数据模型都带有如何使用的设想。有些用法很容易,有些则不支持如此;有些操作运行很快,有些则表现很差;有些数据转换非常自然,有些则很麻烦。

掌握一个数据模型需要花费很多精力(想想关系数据建模有多少本书)。即便只使用一个数据模型,不用操心其内部工作机制,构建软件也是非常困难的。然而,因为数据模型对上层软件的功能(能做什么,不能做什么)有着至深的影响,所以选择一个适合的数据模型是非常重要的。

在本章中,我们将研究一系列用于数据存储和查询的通用数据模型(前面列表中的第 2 点)。特别地,我们将比较关系模型,文档模型和少量基于图形的数据模型。我们还将查看各种查询语言并比较它们的用例。在\autoref{ch:ch3}中,我们将讨论存储引擎是如何工作的。也就是说,这些数据模型实际上是如何实现的(列表中的第 3 点)。

\section{关系模型与文档模型}

现在最著名的数据模型可能是 SQL。它基于 Edgar Codd 在 1970 年提出的关系模型【1】:数据被组织成 \textbf{关系}(SQL 中称作 \textbf{表}),其中每个关系是 \textbf{元组}(SQL 中称作 \textbf{行}) 的无序集合。

关系模型曾是一个理论性的提议,当时很多人都怀疑是否能够有效实现它。然而到了 20 世纪 80 年代中期,关系数据库管理系统(RDBMSes)和 SQL 已成为大多数人们存储和查询某些常规结构的数据的首选工具。关系数据库已经持续称霸了大约 25~30 年 —— 这对计算机史来说是极其漫长的时间。

关系数据库起源于商业数据处理,在 20 世纪 60 年代和 70 年代用大型计算机来执行。从今天的角度来看,那些用例显得很平常:典型的 \textbf{事务处理}(将销售或银行交易,航空公司预订,库存管理信息记录在库)和 \textbf{批处理}(客户发票,工资单,报告)。

当时的其他数据库迫使应用程序开发人员必须考虑数据库内部的数据表示形式。关系模型致力于将上述实现细节隐藏在更简洁的接口之后。

多年来,在数据存储和查询方面存在着许多相互竞争的方法。在 20 世纪 70 年代和 80 年代初,网状模型(network model)和层次模型(hierarchical model)曾是主要的选择,但关系模型(relational model)随后占据了主导地位。对象数据库在 20 世纪 80 年代末和 90 年代初来了又去。XML 数据库在二十一世纪初出现,但只有小众采用过。关系模型的每个竞争者都在其时代产生了大量的炒作,但从来没有持续【2】。

随着电脑越来越强大和互联,它们开始用于日益多样化的目的。关系数据库非常成功地被推广到业务数据处理的原始范围之外更为广泛的用例上。你今天在网上看到的大部分内容依旧是由关系数据库来提供支持,无论是在线发布、讨论、社交网络、电子商务、游戏、软件即服务生产力应用程序等内容。

\subsection{NoSQL 的诞生}

现在——2010 年代,NoSQL 开始了最新一轮尝试,试图推翻关系模型的统治地位。“NoSQL” 这个名字让人遗憾,因为实际上它并没有涉及到任何特定的技术。最初它只是作为一个醒目的 Twitter 标签,用在 2009 年一个关于分布式,非关系数据库上的开源聚会上。无论如何,这个术语触动了某些神经,并迅速在网络创业社区内外传播开来。好些有趣的数据库系统现在都与 \textit{\#NoSQL} 标签相关联,并且 NoSQL 被追溯性地重新解释为 \textbf{不仅是 SQL(Not Only SQL)}【4】。

采用 NoSQL 数据库的背后有几个驱动因素,其中包括:

\begin{itemize}
    \item 需要比关系数据库更好的可伸缩性,包括非常大的数据集或非常高的写入吞吐量
    \item 相比商业数据库产品,免费和开源软件更受偏爱
    \item 关系模型不能很好地支持一些特殊的查询操作
    \item 受挫于关系模型的限制性,渴望一种更具多动态性与表现力的数据模型【5】
\end{itemize}

不同的应用程序有不同的需求,一个用例的最佳技术选择可能不同于另一个用例的最佳技术选择。因此,在可预见的未来,关系数据库似乎可能会继续与各种非关系数据库一起使用——这种想法有时也被称为 \textbf{混合持久化(polyglot persistence)}。

\subsection{对象关系不匹配}

目前大多数应用程序开发都使用面向对象的编程语言来开发,这导致了对 SQL 数据模型的普遍批评:如果数据存储在关系表中,那么需要一个笨拙的转换层,处于应用程序代码中的对象和表,行,列的数据库模型之间。模型之间的不连贯有时被称为\textbf{阻抗不匹配(impedance mismatch)}\footnote{一个从电子学借用的术语。每个电路的输入和输出都有一定的阻抗(交流电阻)。当你将一个电路的输出连接到另一个电路的输入时,如果两个电路的输出和输入阻抗匹配,则连接上的功率传输将被最大化。阻抗不匹配会导致信号反射及其他问题。}。

像 ActiveRecord 和 Hibernate 这样的\textbf{对象关系映射(ORM object-relational mapping)}框架可以减少这个转换层所需的样板代码的数量,但是它们不能完全隐藏这两个模型之间的差异。

\begin{figure}
    \includegraphics[width=0.8\textwidth]{img/fig2-1.png}
    \caption{使用关系型模式来表示领英简介}
    \label{fig:fig2-1}
\end{figure}

例如,\autoref{fig:fig2-1}展示了如何在关系模式中表示简历(一个 LinkedIn 简介)。整个简介可以通过一个唯一的标识符\texttt{user\_id}来标识。像\texttt{first\_name}和\texttt{last\_name}这样的字段每个用户只出现一次,所以可以在 User 表上将其建模为列。但是,大多数人在职业生涯中拥有多于一份的工作,人们可能有不同样的教育阶段和任意数量的联系信息。从用户到这些项目之间存在一对多的关系,可以用多种方式来表示:

\begin{itemize}
    \item 传统 SQL 模型(SQL:1999 之前)中,最常见的规范化表示形式是将职位,教育和联系信息放在单独的表中,对 User 表提供外键引用,如\autoref{fig:fig2-1}所示。
    \item 后续的 SQL 标准增加了对结构化数据类型和 XML 数据的支持;这允许将多值数据存储在单行内,并支持在这些文档内查询和索引。这些功能在 Oracle,IBM DB2,MS SQL Server 和 PostgreSQL 中都有不同程度的支持【6,7】。JSON 数据类型也得到多个数据库的支持,包括 IBM DB2,MySQL 和 PostgreSQL 【8】。
    \item 第三种选择是将职业,教育和联系信息编码为 JSON 或 XML 文档,将其存储在数据库的文本列中,并让应用程序解析其结构和内容。这种配置下,通常不能使用数据库来查询该编码列中的值。
    \item 对于一个像简历这样自包含文档的数据结构而言,JSON 表示是非常合适的:请参阅 [例 2-1]()。JSON 比 XML 更简单。面向文档的数据库(如 MongoDB 【9】,RethinkDB 【10】,CouchDB 【11】和 Espresso【12】)支持这种数据模型。
\end{itemize}

\begin{lstlisting}[caption={用 JSON 文档表示一个 LinkedIn 简介}, breaklines]
    {
      "user_id": 251,
      "first_name": "Bill",
      "last_name": "Gates",
      "summary": "Co-chair of the Bill & Melinda Gates... Active blogger.",
      "region_id": "us:91",
      "industry_id": 131,
      "photo_url": "/p/7/000/253/05b/308dd6e.jpg",
      "positions": [
          {
          "job_title": "Co-chair",
          "organization": "Bill & Melinda Gates Foundation"
          },
          {
          "job_title": "Co-founder, Chairman",
          "organization": "Microsoft"
          }
      ],
      "education": [
          {
          "school_name": "Harvard University",
          "start": 1973,
          "end": 1975
          },
          {
          "school_name": "Lakeside School, Seattle",
          "start": null,
          "end": null
          }
      ],
      "contact_info": {
          "blog": "http://thegatesnotes.com",
          "twitter": "http://twitter.com/BillGates"
      }
    }
\end{lstlisting}

有一些开发人员认为 JSON 模型减少了应用程序代码和存储层之间的阻抗不匹配。不过,正如我们将在\autoref{ch:ch4}中看到的那样,JSON 作为数据编码格式也存在问题。无模式对 JSON 模型来说往往被认为是一个优势;我们将在 “文档模型中的模式灵活性” 中讨论这个问题。

JSON 表示比\autoref{fig:fig2-1}中的多表模式具有更好的\textbf{局部性(locality)}。如果在前面的关系型示例中获取简介,那需要执行多个查询(通过\texttt{user\_id}查询每个表),或者在 User 表与其下属表之间混乱地执行多路连接。而在 JSON 表示中,所有相关信息都在同一个地方,一个查询就足够了。

从用户简介文件到用户职位,教育历史和联系信息,这种一对多关系隐含了数据中的一个树状结构,而 JSON 表示使得这个树状结构变得明确(见\autoref{fig:fig2-2})。

\begin{figure}[htb]
    \includegraphics[width=0.8\textwidth]{img/fig2-2.png}
    \caption{一对多关系构建了一个树结构}
    \label{fig:fig2-2}
\end{figure}

\subsection{多对一和多对多的关系}

在上一节的\author{exa:example2-1}中,\texttt{region\_id} 和 \texttt{industry\_id} 是以 ID,而不是纯字符串 “Greater Seattle Area” 和 “Philanthropy” 的形式给出的。为什么?

如果用户界面用一个自由文本字段来输入区域和行业,那么将他们存储为纯文本字符串是合理的。另一方式是给出地理区域和行业的标准化的列表,并让用户从下拉列表或自动填充器中进行选择,其优势如下:

\begin{itemize}
    \item 各个简介之间样式和拼写统一。
    \item 避免歧义(例如,如果有几个同名的城市)。
    \item 易于更新——名称只存储在一个地方,如果需要更改(例如,由于政治事件而改变城市名称),很容易进行全面更新。
    \item 本地化支持——当网站翻译成其他语言时,标准化的列表可以被本地化,使得地区和行业可以使用用户的语言来显示。
    \item 更好的搜索——例如,搜索华盛顿州的慈善家就会匹配这份简介,因为地区列表可以编码记录西雅图在华盛顿这一事实(从 “Greater Seattle Area” 这个字符串中看不出来)。
\end{itemize}

存储 ID 还是文本字符串,这是个 \textbf{副本(duplication)} 问题。当使用 ID 时,对人类有意义的信息(比如单词:Philanthropy)只存储在一处,所有引用它的地方使用 ID(ID 只在数据库中有意义)。当直接存储文本时,对人类有意义的信息会复制在每处使用记录中。

使用 ID 的好处是,ID 对人类没有任何意义,因而永远不需要改变:ID 可以保持不变,即使它标识的信息发生变化。任何对人类有意义的东西都可能需要在将来某个时候改变 —— 如果这些信息被复制,所有的冗余副本都需要更新。这会导致写入开销,也存在不一致的风险(一些副本被更新了,还有些副本没有被更新)。去除此类重复是数据库 \textbf{规范化(normalization)} 的关键思想。\footnote{关于关系模型的文献区分了几种不同的规范形式,但这些区别几乎没有实际意义。一个经验法则是,如果重复存储了可以存储在一个地方的值,则模式就不是 \textbf{规范化(normalized)} 的。}

\begin{quote}
    数据库管理员和开发人员喜欢争论规范化和非规范化,让我们暂时保留判断吧。在本书的\autoref{part:part-3},我们将回到这个话题,探讨系统的方法用以处理缓存,非规范化和衍生数据。
\end{quote}

不幸的是,对这些数据进行规范化需要多对一的关系(许多人生活在一个特定的地区,许多人在一个特定的行业工作),这与文档模型不太吻合。在关系数据库中,通过 ID 来引用其他表中的行是正常的,因为连接很容易。在文档数据库中,一对多树结构没有必要用连接,对连接的支持通常很弱\footnote{在撰写本文时,RethinkDB 支持连接,MongoDB 不支持连接,而 CouchDB 只支持预先声明的视图。}。

如果数据库本身不支持连接,则必须在应用程序代码中通过对数据库进行多个查询来模拟连接。(在这种情况中,地区和行业的列表可能很小,改动很少,应用程序可以简单地将其保存在内存中。不过,执行连接的工作从数据库被转移到应用程序代码上。)

此外,即便应用程序的最初版本适合无连接的文档模型,随着功能添加到应用程序中,数据会变得更加互联。例如,考虑一下对简历例子进行的一些修改:

\begin{itemize}
    \item 组织和学校作为实体
          在前面的描述中,\texttt{organization}(用户工作的公司)和 \texttt{school\_name}(他们学习的地方)只是字符串。也许他们应该是对实体的引用呢?然后,每个组织、学校或大学都可以拥有自己的网页(标识、新闻提要等)。每个简历可以链接到它所提到的组织和学校,并且包括他们的图标和其他信息(请参阅\autoref{fig:fig2-3},来自 LinkedIn 的一个例子)。
    \item 推荐
          假设你想添加一个新的功能:一个用户可以为另一个用户写一个推荐。在用户的简历上显示推荐,并附上推荐用户的姓名和照片。如果推荐人更新他们的照片,那他们写的任何推荐都需要显示新的照片。因此,推荐应该拥有作者个人简介的引用。
\end{itemize}

\begin{figure}
    \centering
    \includegraphics[width=0.8\textwidth]{img/fig2-3.png}
    \caption{公司名不仅是字符串,还是一个指向公司实体的链接(LinkedIn 截图)}
    \label{fig:fig2-3}
\end{figure}

\autoref{fig:fig2-4}阐明了这些新功能需要如何使用多对多关系。每个虚线矩形内的数据可以分组成一个文档,但是对单位,学校和其他用户的引用需要表示成引用,并且在查询时需要连接。

\begin{figure}
    \centering
    \includegraphics[width=0.8\textwidth]{img/fig2-4.png}
    \caption{使用多对多关系扩展简历}
    \label{fig:fig2-4}
\end{figure}

\subsection{文档数据库是否在重蹈覆辙?}

在多对多的关系和连接已常规用在关系数据库时,文档数据库和 NoSQL 重启了辩论:如何以最佳方式在数据库中表示多对多关系。那场辩论可比 NoSQL 古老得多,事实上,最早可以追溯到计算机化数据库系统。

20 世纪 70 年代最受欢迎的业务数据处理数据库是 IBM 的信息管理系统(IMS),最初是为了阿波罗太空计划的库存管理而开发的,并于 1968 年有了首次商业发布【13】。目前它仍在使用和维护,运行在 IBM 大型机的 OS/390 上【14】。

IMS 的设计中使用了一个相当简单的数据模型,称为\textbf{层次模型(hierarchical model)},它与文档数据库使用的 JSON 模型有一些惊人的相似之处【2】。它将所有数据表示为嵌套在记录中的记录树,这很像\autoref{fig:fig2-2}的 JSON 结构。

同文档数据库一样,IMS 能良好处理一对多的关系,但是很难应对多对多的关系,并且不支持连接。开发人员必须决定是否复制(非规范化)数据或手动解决从一个记录到另一个记录的引用。这些二十世纪六七十年代的问题与现在开发人员遇到的文档数据库问题非常相似【15】。

那时人们提出了各种不同的解决方案来解决层次模型的局限性。其中最突出的两个是\textbf{关系模型}(relational model,它变成了 SQL,并统治了世界)和 \textbf{网状模型}(network model,最初很受关注,但最终变得冷门)。这两个阵营之间的 “大辩论” 在 70 年代持续了很久时间【2】。

那两个模式解决的问题与当前的问题相关,因此值得简要回顾一下那场辩论。

\subsubsection{网状模型}

网状模型由一个称为数据系统语言会议(CODASYL)的委员会进行了标准化,并被数个不同的数据库厂商实现;它也被称为 CODASYL 模型【16】。

CODASYL 模型是层次模型的推广。在层次模型的树结构中,每条记录只有一个父节点;在网络模式中,每条记录可能有多个父节点。例如,“Greater Seattle Area” 地区可能是一条记录,每个居住在该地区的用户都可以与之相关联。这允许对多对一和多对多的关系进行建模。

网状模型中记录之间的链接不是外键,而更像编程语言中的指针(同时仍然存储在磁盘上)。访问记录的唯一方法是跟随从根记录起沿这些链路所形成的路径。这被称为 \textbf{访问路径(access path)}。

最简单的情况下,访问路径类似遍历链表:从列表头开始,每次查看一条记录,直到找到所需的记录。但在多对多关系的情况中,数条不同的路径可以到达相同的记录,网状模型的程序员必须跟踪这些不同的访问路径。

CODASYL 中的查询是通过利用遍历记录列和跟随访问路径表在数据库中移动游标来执行的。如果记录有多个父结点(即多个来自其他记录的传入指针),则应用程序代码必须跟踪所有的各种关系。甚至 CODASYL 委员会成员也承认,这就像在 n 维数据空间中进行导航【17】。

尽管手动选择访问路径能够最有效地利用 20 世纪 70 年代非常有限的硬件功能(如磁带驱动器,其搜索速度非常慢),但这使得查询和更新数据库的代码变得复杂不灵活。无论是分层还是网状模型,如果你没有所需数据的路径,就会陷入困境。你可以改变访问路径,但是必须浏览大量手写数据库查询代码,并重写来处理新的访问路径。更改应用程序的数据模型是很难的。

\subsubsection{关系模型}

相比之下,关系模型做的就是将所有的数据放在光天化日之下:一个 \textbf{关系(表)} 只是一个 \textbf{元组(行)} 的集合,仅此而已。如果你想读取数据,它没有迷宫似的嵌套结构,也没有复杂的访问路径。你可以选中符合任意条件的行,读取表中的任何或所有行。你可以通过指定某些列作为匹配关键字来读取特定行。你可以在任何表中插入一个新的行,而不必担心与其他表的外键关系\footnote{外键约束允许对修改进行限制,但对于关系模型这并不是必选项。即使有约束,外键连接在查询时执行,而在 CODASYL 中,连接在插入时高效完成。}。

在关系数据库中,查询优化器自动决定查询的哪些部分以哪个顺序执行,以及使用哪些索引。这些选择实际上是 “访问路径”,但最大的区别在于它们是由查询优化器自动生成的,而不是由程序员生成,所以我们很少需要考虑它们。

如果想按新的方式查询数据,你可以声明一个新的索引,查询会自动使用最合适的那些索引。无需更改查询来利用新的索引(请参阅 “数据查询语言”)。关系模型因此使添加应用程序新功能变得更加容易。

关系数据库的查询优化器是复杂的,已耗费了多年的研究和开发精力【18】。关系模型的一个关键洞察是:只需构建一次查询优化器,随后使用该数据库的所有应用程序都可以从中受益。如果你没有查询优化器的话,那么为特定查询手动编写访问路径比编写通用优化器更容易——不过从长期看通用解决方案更好。

\subsection{关系型数据库与文档数据库在今日的对比}

将关系数据库与文档数据库进行比较时,可以考虑许多方面的差异,包括它们的容错属性(请参阅\autoref{ch:ch5})和处理并发性(请参阅\autoref{ch:ch7})。本章将只关注数据模型中的差异。

支持文档数据模型的主要论据是架构灵活性,因局部性而拥有更好的性能,以及对于某些应用程序而言更接近于应用程序使用的数据结构。关系模型通过为连接提供更好的支持以及支持多对一和多对多的关系来反击。

\subsubsection{哪种数据模型更有助于简化应用代码?}

如果应用程序中的数据具有类似文档的结构(即,一对多关系树,通常一次性加载整个树),那么使用文档模型可能是一个好主意。将类似文档的结构分解成多个表(如\autoref{fig:fig2-1}中的 \texttt{positions}、\texttt{education} 和 \texttt{contact\_info})的关系技术可能导致繁琐的模式和不必要的复杂的应用程序代码。

文档模型有一定的局限性:例如,不能直接引用文档中的嵌套的项目,而是需要说 “用户 251 的位置列表中的第二项”(很像层次模型中的访问路径)。但是,只要文件嵌套不太深,这通常不是问题。

文档数据库对连接的糟糕支持可能是个问题,也可能不是问题,这取决于应用程序。例如,如果某分析型应用程序使用一个文档数据库来记录何时何地发生了何事,那么多对多关系可能永远也用不上。【19】。

但如果你的应用程序确实会用到多对多关系,那么文档模型就没有那么诱人了。尽管可以通过反规范化来消除对连接的需求,但这需要应用程序代码来做额外的工作以确保数据一致性。尽管应用程序代码可以通过向数据库发出多个请求的方式来模拟连接,但这也将复杂性转移到应用程序中,而且通常也会比由数据库内的专用代码更慢。在这种情况下,使用文档模型可能会导致更复杂的应用代码与更差的性能【15】。

我们没有办法说哪种数据模型更有助于简化应用代码,因为它取决于数据项之间的关系种类。对高度关联的数据而言 糟糕的,关系模型是可以接受的,而选用图形模型(请参阅 “图数据模型”)是最自然的。

\subsubsection{文档模型中的模式灵活性}

大多数文档数据库以及关系数据库中的 JSON 支持都不会强制文档中的数据采用何种模式。关系数据库的 XML 支持通常带有可选的模式验证。没有模式意味着可以将任意的键和值添加到文档中,并且当读取时,客户端无法保证文档可能包含的字段。

文档数据库有时称为 \textbf{无模式(schemaless)},但这具有误导性,因为读取数据的代码通常假定某种结构 —— 即存在隐式模式,但不由数据库强制执行【20】。一个更精确的术语是 \textbf{读时模式}(即 schema-on-read,数据的结构是隐含的,只有在数据被读取时才被解释),相应的是 \textbf{写时模式}(即 schema-on-write,传统的关系数据库方法中,模式明确,且数据库确保所有的数据都符合其模式)【21】。

读时模式类似于编程语言中的动态(运行时)类型检查,而写时模式类似于静态(编译时)类型检查。就像静态和动态类型检查的相对优点具有很大的争议性一样【22】,数据库中模式的强制性是一个具有争议的话题,一般来说没有正确或错误的答案。

在应用程序想要改变其数据格式的情况下,这些方法之间的区别尤其明显。例如,假设你把每个用户的全名存储在一个字段中,而现在想分别存储名字和姓氏【23】。在文档数据库中,只需开始写入具有新字段的新文档,并在应用程序中使用代码来处理读取旧文档的情况。例如:

\begin{lstlisting}
if (user && user.name && !user.first_name) {
    // Documents written before Dec 8, 2013 don't have first_name
    user.first_name = user.name.split(" ")[0];
}
\end{lstlisting}

另一方面,在 “静态类型” 数据库模式中,通常会执行以下 \textbf{迁移(migration)} 操作:

\begin{lstlisting}
ALTER TABLE users ADD COLUMN first_name text;
UPDATE users SET first_name = split_part(name, ' ', 1);      -- PostgreSQL
UPDATE users SET first_name = substring_index(name, ' ', 1);      -- MySQL
\end{lstlisting}

模式变更的速度很慢,而且要求停运。它的这种坏名誉并不是完全应得的:大多数关系数据库系统可在几毫秒内执行 \texttt{ALTER TABLE} 语句。MySQL 是一个值得注意的例外,它执行 \texttt{ALTER TABLE} 时会复制整个表,这可能意味着在更改一个大型表时会花费几分钟甚至几个小时的停机时间,尽管存在各种工具来解决这个限制【24,25,26】。

大型表上运行 \texttt{UPDATE} 语句在任何数据库上都可能会很慢,因为每一行都需要重写。要是不可接受的话,应用程序可以将 \texttt{first\_name} 设置为默认值 \texttt{NULL},并在读取时再填充,就像使用文档数据库一样。

当由于某种原因(例如,数据是异构的)集合中的项目并不都具有相同的结构时,读时模式更具优势。例如,如果:

\begin{itemize}
    \item 存在许多不同类型的对象,将每种类型的对象放在自己的表中是不现实的。
    \item 数据的结构由外部系统决定。你无法控制外部系统且它随时可能变化。
\end{itemize}

在上述情况下,模式的坏处远大于它的帮助,无模式文档可能是一个更加自然的数据模型。但是,要是所有记录都具有相同的结构,那么模式是记录并强制这种结构的有效机制。第四章将更详细地讨论模式和模式演化。

\subsubsection{查询的数据局部性}

文档通常以单个连续字符串形式进行存储,编码为 JSON、XML 或其二进制变体(如 MongoDB 的 BSON)。如果应用程序经常需要访问整个文档(例如,将其渲染至网页),那么存储局部性会带来性能优势。如果将数据分割到多个表中(如 [图 2-1](img/fig2-1.png) 所示),则需要进行多次索引查找才能将其全部检索出来,这可能需要更多的磁盘查找并花费更多的时间。

局部性仅仅适用于同时需要文档绝大部分内容的情况。即使只访问文档其中的一小部分,数据库通常需要加载整个文档,对于大型文档来说这种加载行为是很浪费的。更新文档时,通常需要整个重写。只有不改变文档大小的修改才可以容易地原地执行。因此,通常建议保持相对小的文档,并避免增加文档大小的写入【9】。这些性能限制大大减少了文档数据库的实用场景。

值得指出的是,为了局部性而分组集合相关数据的想法并不局限于文档模型。例如,Google 的 Spanner 数据库在关系数据模型中提供了同样的局部性属性,允许模式声明一个表的行应该交错(嵌套)在父表内【27】。Oracle 类似地允许使用一个称为 \textbf{多表索引集群表(multi-table index cluster tables)} 的类似特性【28】。Bigtable 数据模型(用于 Cassandra 和 HBase)中的 \textbf{列族(column-family)} 概念与管理局部性的目的类似【29】。

在\autoref{ch:ch3}将还会看到更多关于局部性的内容。

\subsubsection{文档和关系数据库的融合}

自 2000 年代中期以来,大多数关系数据库系统(MySQL 除外)都已支持 XML。这包括对 XML 文档进行本地修改的功能,以及在 XML 文档中进行索引和查询的功能。这允许应用程序使用那种与文档数据库应当使用的非常类似的数据模型。

从 9.3 版本开始的 PostgreSQL 【8】,从 5.7 版本开始的 MySQL 以及从版本 10.5 开始的 IBM DB2【30】也对 JSON 文档提供了类似的支持级别。鉴于用在 Web APIs 的 JSON 流行趋势,其他关系数据库很可能会跟随他们的脚步并添加 JSON 支持。

在文档数据库中,RethinkDB 在其查询语言中支持类似关系的连接,一些 MongoDB 驱动程序可以自动解析数据库引用(有效地执行客户端连接,尽管这可能比在数据库中执行的连接慢,需要额外的网络往返,并且优化更少)。

随着时间的推移,关系数据库和文档数据库似乎变得越来越相似,这是一件好事:数据模型相互补充\footnote{Codd 对关系模型【1】的原始描述实际上允许在关系模式中与 JSON 文档非常相似。他称之为 \textbf{非简单域(nonsimple domains)}。这个想法是,一行中的值不一定是一个像数字或字符串一样的原始数据类型,也可以是一个嵌套的关系(表),因此可以把一个任意嵌套的树结构作为一个值,这很像 30 年后添加到 SQL 中的 JSON 或 XML 支持。},如果一个数据库能够处理类似文档的数据,并能够对其执行关系查询,那么应用程序就可以使用最符合其需求的功能组合。

关系模型和文档模型的混合是未来数据库一条很好的路线。

\section{数据查询语言}

当引入关系模型时,关系模型包含了一种查询数据的新方法:SQL 是一种 \textbf{声明式} 查询语言,而 IMS 和 CODASYL 使用 \textbf{命令式} 代码来查询数据库。那是什么意思?

许多常用的编程语言是命令式的。例如,给定一个动物物种的列表,返回列表中的鲨鱼可以这样写:

\begin{lstlisting}
function getSharks() {
    var sharks = [];
    for (var i = 0; i < animals.length; i++) {
        if (animals[i].family === "Sharks") {
        sharks.push(animals[i]);
        }
    }
    return sharks;
}
\end{lstlisting}

而在关系代数中,你可以这样写:

\[
    sharks = \sigma_{family = "sharks"}(animals)
\]

其中 $\sigma$(希腊字母西格玛)是选择操作符,只返回符合 \texttt{family="shark"} 条件的动物。

定义 SQL 时,它紧密地遵循关系代数的结构:

\begin{lstlisting}
    SELECT * FROM animals WHERE family ='Sharks';
\end{lstlisting}

命令式语言告诉计算机以特定顺序执行某些操作。可以想象一下,逐行地遍历代码,评估条件,更新变量,并决定是否再循环一遍。

在声明式查询语言(如 SQL 或关系代数)中,你只需指定所需数据的模式——结果必须符合哪些条件,以及如何将数据转换(例如,排序,分组和集合)——但不是如何实现这一目标。数据库系统的查询优化器决定使用哪些索引和哪些连接方法,以及以何种顺序执行查询的各个部分。

声明式查询语言是迷人的,因为它通常比命令式 API 更加简洁和容易。但更重要的是,它还隐藏了数据库引擎的实现细节,这使得数据库系统可以在无需对查询做任何更改的情况下进行性能提升。

例如,在本节开头所示的命令代码中,动物列表以特定顺序出现。如果数据库想要在后台回收未使用的磁盘空间,则可能需要移动记录,这会改变动物出现的顺序。数据库能否安全地执行,而不会中断查询?

SQL 示例不确保任何特定的顺序,因此不在意顺序是否改变。但是如果查询用命令式的代码来写的话,那么数据库就永远不可能确定代码是否依赖于排序。SQL 相当有限的功能性为数据库提供了更多自动优化的空间。

最后,声明式语言往往适合并行执行。现在,CPU 的速度通过核心(core)的增加变得更快,而不是以比以前更高的时钟速度运行【31】。命令代码很难在多个核心和多个机器之间并行化,因为它指定了指令必须以特定顺序执行。声明式语言更具有并行执行的潜力,因为它们仅指定结果的模式,而不指定用于确定结果的算法。在适当情况下,数据库可以自由使用查询语言的并行实现【32】。

\subsection{Web 上的声明式查询}

声明式查询语言的优势不仅限于数据库。为了说明这一点,让我们在一个完全不同的环境中比较声明式和命令式方法:一个 Web 浏览器。

假设你有一个关于海洋动物的网站。用户当前正在查看鲨鱼页面,因此你将当前所选的导航项目 “鲨鱼” 标记为当前选中项目。

\begin{lstlisting}
<ul>
  <li class="selected">
    <p>Sharks</p>
    <ul>
      <li>Great White Shark</li>
      <li>Tiger Shark</li>
      <li>Hammerhead Shark</li>
    </ul>
  </li>
  <li>
    <p>Whales</p>
    <ul>
      <li>Blue Whale</li>
      <li>Humpback Whale</li>
      <li>Fin Whale</li>
    </ul>
  </li>
</ul>
\end{lstlisting}

现在想让当前所选页面的标题具有一个蓝色的背景,以便在视觉上突出显示。使用 CSS 实现起来非常简单:

\begin{lstlisting}
li.selected > p {
    background-color: blue;
}   
\end{lstlisting}

这里的 CSS 选择器 \texttt{li.selected > p} 声明了我们想要应用蓝色样式的元素的模式:即其直接父元素是具有 CSS 类 \texttt{selected} 的 \texttt{<li>} 元素的所有 \texttt{<p>} 元素。示例中的元素 \texttt{<p>Sharks</p>} 匹配此模式,但 \texttt{<p>Whales</p>} 不匹配,因为其 \texttt{<li>} 父元素缺少 \texttt{class="selected"}。

如果使用 XSL 而不是 CSS,你可以做类似的事情:

\begin{lstlisting}
<xsl:template match="li[@class='selected']/p">
<fo:block background-color="blue">
<xsl:apply-templates/>
</fo:block>
</xsl:template>
\end{lstlisting}

这里的 XPath 表达式 \texttt{li[@class='selected']/p} 相当于上例中的 CSS 选择器 \texttt{li.selected > p}。CSS 和 XSL 的共同之处在于,它们都是用于指定文档样式的声明式语言。

想象一下,必须使用命令式方法的情况会是如何。在 Javascript 中,使用 \textbf{文档对象模型(DOM)} API,其结果可能如下所示:

\begin{lstlisting}
var liElements = document.getElementsByTagName("li");
for (var i = 0; i < liElements.length; i++) {
    if (liElements[i].className === "selected") {
        var children = liElements[i].childNodes;
        for (var j = 0; j < children.length; j++) {
            var child = children[j];
            if (child.nodeType === Node.ELEMENT_NODE && child.tagName === "P") {
                child.setAttribute("style", "background-color: blue");
            }
        }
    }
}
\end{lstlisting}

这段 JavaScript 代码命令式地将元素设置为蓝色背景,但是代码看起来很糟糕。不仅比 CSS 和 XSL 等价物更长,更难理解,而且还有一些严重的问题:

\begin{itemize}
    \item 如果选定的类被移除(例如,因为用户点击了不同的页面),即使代码重新运行,蓝色背景也不会被移除 - 因此该项目将保持突出显示,直到整个页面被重新加载。使用 CSS,浏览器会自动检测 \texttt{li.selected > p} 规则何时不再适用,并在选定的类被移除后立即移除蓝色背景。
    \item 如果你想要利用新的 API(例如 \texttt{document.getElementsByClassName("selected")} 甚至 \texttt{document.evaluate()})来提高性能,则必须重写代码。另一方面,浏览器供应商可以在不破坏兼容性的情况下提高 CSS 和 XPath 的性能。
\end{itemize}

在 Web 浏览器中,使用声明式 CSS 样式比使用 JavaScript 命令式地操作样式要好得多。类似地,在数据库中,使用像 SQL 这样的声明式查询语言比使用命令式查询 API 要好得多\footnote{IMS 和 CODASYL 都使用命令式 API。应用程序通常使用 COBOL 代码遍历数据库中的记录,一次一条记录【2,16】。}。

\subsection{MapReduce 查询}

MapReduce 是一个由 Google 推广的编程模型,用于在多台机器上批量处理大规模的数据【33】。一些 NoSQL 数据存储(包括 MongoDB 和 CouchDB)支持有限形式的 MapReduce,作为在多个文档中执行只读查询的机制。

关于 MapReduce 更详细的介绍在\autoref{ch:ch10}。现在我们只简要讨论一下 MongoDB 使用的模型。

MapReduce 既不是一个声明式的查询语言,也不是一个完全命令式的查询 API,而是处于两者之间:查询的逻辑用代码片段来表示,这些代码片段会被处理框架重复性调用。它基于 \texttt{map}(也称为 \texttt{collect})和 \texttt{reduce}(也称为 \texttt{fold} 或 \texttt{inject})函数,两个函数存在于许多函数式编程语言中。

最好举例来解释 MapReduce 模型。假设你是一名海洋生物学家,每当你看到海洋中的动物时,你都会在数据库中添加一条观察记录。现在你想生成一个报告,说明你每月看到多少鲨鱼。

在 PostgreSQL 中,你可以像这样表述这个查询:

\begin{lstlisting}
SELECT
date_trunc('month', observation_timestamp) AS observation_month,
sum(num_animals)                           AS total_animals
FROM observations
WHERE family = 'Sharks'
GROUP BY observation_month;
\end{lstlisting}

\texttt{date\_trunc('month', timestamp)} 函数用于确定包含 \texttt{timestamp} 的日历月份,并返回代表该月份开始的另一个时间戳。换句话说,它将时间戳舍入成最近的月份。

这个查询首先过滤观察记录,以只显示鲨鱼家族的物种,然后根据它们发生的日历月份对观察记录果进行分组,最后将在该月的所有观察记录中看到的动物数目加起来。

同样的查询用 MongoDB 的 MapReduce 功能可以按如下来表述:

\begin{lstlisting}
db.observations.mapReduce(
    function map() {
        var year = this.observationTimestamp.getFullYear();
        var month = this.observationTimestamp.getMonth() + 1;
        emit(year + "-" + month, this.numAnimals);
    },
    function reduce(key, values) {
        return Array.sum(values);
    },
    {
        query: {
            family: "Sharks",
        },
        out: "monthlySharkReport",
    }
); 
\end{lstlisting}

\begin{itemize}
    \item 可以声明式地指定一个只考虑鲨鱼种类的过滤器(这是 MongoDB 特定的 MapReduce 扩展)。
    \item 每个匹配查询的文档都会调用一次 JavaScript 函数 \texttt{map},将 \texttt{this} 设置为文档对象。
    \item \texttt{map} 函数发出一个键(包括年份和月份的字符串,如 \texttt{"2013-12"} 或 \texttt{"2014-1"})和一个值(该观察记录中的动物数量)。
    \item \texttt{map} 发出的键值对按键来分组。对于具有相同键(即,相同的月份和年份)的所有键值对,调用一次 \texttt{reduce} 函数。
    \item \texttt{reduce} 函数将特定月份内所有观测记录中的动物数量相加。
    \item 将最终的输出写入到 \texttt{monthlySharkReport} 集合中。
\end{itemize}

例如,假设 \texttt{observations} 集合包含这两个文档:

\begin{lstlisting}
{
    observationTimestamp: Date.parse(  "Mon, 25 Dec 1995 12:34:56 GMT"),
    family: "Sharks",
    species: "Carcharodon carcharias",
    numAnimals: 3
}
{
    observationTimestamp: Date.parse("Tue, 12 Dec 1995 16:17:18 GMT"),
    family: "Sharks",
    species:    "Carcharias taurus",
    numAnimals: 4
}
\end{lstlisting}

对每个文档都会调用一次 \texttt{map} 函数,结果将是 \texttt{emit("1995-12",3)} 和 \texttt{emit("1995-12",4)}。随后,以 \texttt{reduce("1995-12",[3,4])} 调用 \texttt{reduce} 函数,将返回 \texttt{7}。

map 和 reduce 函数在功能上有所限制:它们必须是 \textbf{纯} 函数,这意味着它们只使用传递给它们的数据作为输入,它们不能执行额外的数据库查询,也不能有任何副作用。这些限制允许数据库以任何顺序运行任何功能,并在失败时重新运行它们。然而,map 和 reduce 函数仍然是强大的:它们可以解析字符串、调用库函数、执行计算等等。

MapReduce 是一个相当底层的编程模型,用于计算机集群上的分布式执行。像 SQL 这样的更高级的查询语言可以用一系列的 MapReduce 操作来实现(见 [第十章](ch10.md)),但是也有很多不使用 MapReduce 的分布式 SQL 实现。须注意,SQL 并没有限制它只能在单一机器上运行,而 MapReduce 也并没有垄断所有的分布式查询执行。

能够在查询中使用 JavaScript 代码是高级查询的一个重要特性,但这不限于 MapReduce,一些 SQL 数据库也可以用 JavaScript 函数进行扩展【34】。

MapReduce 的一个可用性问题是,必须编写两个密切合作的 JavaScript 函数,这通常比编写单个查询更困难。此外,声明式查询语言为查询优化器提供了更多机会来提高查询的性能。基于这些原因,MongoDB 2.2 添加了一种叫做 \textbf{聚合管道} 的声明式查询语言的支持【9】。用这种语言表述鲨鱼计数查询如下所示:

\begin{lstlisting}
    db.observations.aggregate([
        { $match: { family: "Sharks" } },
        {
          $group: {
            _id: {
              year: { $year: "$observationTimestamp" },
              month: { $month: "$observationTimestamp" },
            },
            totalAnimals: { $sum: "$numAnimals" },
          },
        },
      ]);
\end{lstlisting}

聚合管道语言的表现力与(前述 PostgreSQL 例子的)SQL 子集相当,但是它使用基于 JSON 的语法而不是 SQL 那种接近英文句式的语法;这种差异也许只是口味问题。这个故事的寓意是:NoSQL 系统可能会意外发现自己只是重新发明了一套经过乔装改扮的 SQL。

\newpage
\begin{figure}
  \centering
  \includegraphics[width=\textwidth]{img/ch3.png}
  \label{fig:ch3}
\end{figure}

\chapter{存储与检索}
\label{ch:ch3}
\include{content/ch3}

\newpage
\begin{figure}
  \centering
  \includegraphics[width=\textwidth]{img/ch4.png}
  \label{fig:ch4}
\end{figure}

\chapter{编码与演化}
\label{ch:ch4}

\part{分布式数据}
\label{part:part-2}

\newpage
\begin{figure}
  \centering
  \includegraphics[width=\textwidth]{img/ch5.png}
  \label{fig:ch5}
\end{figure}

\chapter{复制}
\label{ch:ch5}

\newpage
\begin{figure}
  \centering
  \includegraphics[width=\textwidth]{img/ch6.png}
  \label{fig:ch6}
\end{figure}

\chapter{分区}
\label{ch:ch6}

\newpage
\begin{figure}
  \centering
  \includegraphics[width=\textwidth]{img/ch7.png}
  \label{fig:ch7}
\end{figure}

\chapter{事务}
\label{ch:ch7}

\newpage
\begin{figure}
  \centering
  \includegraphics[width=\textwidth]{img/ch8.png}
  \label{fig:ch8}
\end{figure}

\chapter{分布式系统的麻烦}
\label{ch:ch8}

\newpage
\begin{figure}
  \centering
  \includegraphics[width=\textwidth]{img/ch9.png}
  \label{fig:ch9}
\end{figure}

\chapter{一致性与共识}
\label{ch:ch9}

\part{衍生数据}
\label{part:part-3}

\newpage
\begin{figure}
  \centering
  \includegraphics[width=\textwidth]{img/ch10.png}
  \label{fig:ch10}
\end{figure}

\chapter{批处理}
\label{ch:ch10}

\newpage
\begin{figure}
  \centering
  \includegraphics[width=\textwidth]{img/ch11.png}
  \label{fig:ch11}
\end{figure}

\chapter{流处理}
\label{ch:ch11}

\newpage
\begin{figure}
  \centering
  \includegraphics[width=\textwidth]{img/ch12.png}
  \label{fig:ch12}
\end{figure}

\chapter{数据系统的未来}
\label{ch:ch12}

\end{document}